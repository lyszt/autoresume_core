% Section: Projects (from repos or config; no tags)
\sectionTitle{ Projekter }{\faCodeFork}

\begin{projects}
\begin{minipage}[t]{\dimexpr(\linewidth) - 1.5em}
  \textbf{\textsc{ Autosume_core }} \website{ https://github.com/lyszt/autoresume_core }{ Autosume_core } \hfill \textsc{ Core }\smallskip\\
  
\end{minipage}
\\*
\begin{minipage}[t]{\dimexpr(\linewidth) - 1.5em}
  \textbf{\textsc{ Le capitalisme-automatique_core }} \website{ https://github.com/lyszt/capitalisme-automatique_core }{ Le capitalisme-automatique_core } \hfill \textsc{ Core }\smallskip\\
  Suite à un simulateur de société java fait pendant une classe de Java, mais en Python
\end{minipage}
\\*
\begin{minipage}[t]{\dimexpr(\linewidth) - 1.5em}
  \textbf{\textsc{ clairemont_core }} \website{ https://github.com/lyszt/clairemont_core }{ clairemont_core } \hfill \textsc{ Core }\smallskip\\
  
\end{minipage}
\\*
\begin{minipage}[t]{\dimexpr(\linewidth) - 1.5em}
  \textbf{\textsc{ dannazione-di-provvidenza_core }} \website{ https://github.com/lyszt/dannazione-di-provvidenza_core }{ dannazione-di-provvidenza_core } \hfill \textsc{ Core }\smallskip\\
  Dannazione di Providenza est un assistant d'apprentissage des langues alimenté par l'IA avec des capacités OCR, des services de traduction et une extension de navigateur Firefox.
\end{minipage}
\\*
\begin{minipage}[t]{\dimexpr(\linewidth) - 1.5em}
  \textbf{\textsc{ - Le navire français-Madame-de-pompadour_core }} \website{ https://github.com/lyszt/french-ship-madame-de-pompadour_core }{ - Le navire français-Madame-de-pompadour_core } \hfill \textsc{ Core }\smallskip\\
  Simulation sociale et générateur d'histoires qui crée des récits de science-fiction basés sur Star Trek. OOP Python / Flask avec certaines structures de données, Gemini API, React / Vite / Tailwind frontend.
\end{minipage}
\\*
\begin{minipage}[t]{\dimexpr(\linewidth) - 1.5em}
  \textbf{\textsc{ iris-client_core }} \website{ https://github.com/lyszt/iris-client_core }{ iris-client_core } \hfill \textsc{ Core }\smallskip\\
  Interface utilisateur du terminal conçue pour le projet de gestion de projet IRIS.
\end{minipage}
\\*
\begin{minipage}[t]{\dimexpr(\linewidth) - 1.5em}
  \textbf{\textsc{ le cœur }} \website{ https://github.com/lyszt/kyrie_core }{ le cœur } \hfill \textsc{ Core }\smallskip\\
  
\end{minipage}
\\*
\begin{minipage}[t]{\dimexpr(\linewidth) - 1.5em}
  \textbf{\textsc{ providence_core }} \website{ https://github.com/lyszt/providence_core }{ providence_core } \hfill \textsc{ Core }\smallskip\\
  Réseau de Providentia réfactorisé.
\end{minipage}
\\*
\begin{minipage}[t]{\dimexpr(\linewidth) - 1.5em}
  \textbf{\textsc{ providentia-access_core }} \website{ https://github.com/lyszt/providentia-access_core }{ providentia-access_core } \hfill \textsc{ Core }\smallskip\\
  Le gestionnaire de mot de passe et d'authentification de Providentia.
\end{minipage}
\\*
\begin{minipage}[t]{\dimexpr(\linewidth) - 1.5em}
  \textbf{\textsc{ La première partie de la série }} \website{ https://github.com/lyszt/providentia-magnata_core }{ La première partie de la série } \hfill \textsc{ Core }\smallskip\\
  Providence est un bot Discord conçu pour analyser les messages des utilisateurs, gérer les données des utilisateurs et améliorer l'engagement grâce à la traduction linguistique et à l'analyse des sentiments.
\end{minipage}
\\*
\begin{minipage}[t]{\dimexpr(\linewidth) - 1.5em}
  \textbf{\textsc{ Ryujinni_core }} \website{ https://github.com/lyszt/ryujinni_core }{ Ryujinni_core } \hfill \textsc{ Core }\smallskip\\
  La version Elixir du robot Clairemont de Discord, alimenté par le réseau Providence (Providentia).
\end{minipage}
\\*
\begin{minipage}[t]{\dimexpr(\linewidth) - 1.5em}
  \textbf{\textsc{ la ville de Scarlett-Citadel_core }} \website{ https://github.com/lyszt/scarlett-citadel_core }{ la ville de Scarlett-Citadel_core } \hfill \textsc{ Core }\smallskip\\
  La citadelle Scarlett, la version révisée de mon site Web personnel, compilée à l'aide de vite, en un site Web statique, en fait, elle a été changée en Tesserae.
\end{minipage}
\\*
\begin{minipage}[t]{\dimexpr(\linewidth) - 1.5em}
  \textbf{\textsc{ nombre de points }} \website{ https://github.com/lyszt/talleyrand_core }{ nombre de points } \hfill \textsc{ Core }\smallskip\\
  .config Linux configuration manager construit en C++.
\end{minipage}
\\*
\begin{minipage}[t]{\dimexpr(\linewidth) - 1.5em}
  \textbf{\textsc{ à la base }} \website{ https://github.com/lyszt/tesserae_core }{ à la base } \hfill \textsc{ Core }\smallskip\\
  Une réflexion sur le cat lyszt.github.io avec un serveur pour l'engagement communautaire.
\end{minipage}
\\*
\begin{minipage}[t]{\dimexpr(\linewidth) - 1.5em}
  \textbf{\textsc{ travail-ed2_edu }} \website{ https://github.com/lyszt/trabalho-ed2_edu }{ travail-ed2_edu } \hfill \textsc{ Education }\smallskip\\
  
\end{minipage}
\\*
\begin{minipage}[t]{\dimexpr(\linewidth) - 1.5em}
  \textbf{\textsc{ travail_edu }} \website{ https://github.com/lyszt/workonomia_edu }{ travail_edu } \hfill \textsc{ Education }\smallskip\\
  Workonomia est un simulateur de gestion économique développé en Java à l'aide de la bibliothèque Swing, axé sur la gestion des ressources et l'optimisation du système.
\end{minipage}
\\*
\begin{minipage}[t]{\dimexpr(\linewidth) - 1.5em}
  \textbf{\textsc{ Kairos-praxis_héritage }} \website{ https://github.com/lyszt/kairos-praxis_legacy }{ Kairos-praxis_héritage } \hfill \textsc{ Legacy }\smallskip\\
  Kairos Praxis (Insight 2) offre des fonctionnalités d'analyse des concurrents via une application de bureau. Le grattage Web automatisé permet de collecter des données relatives aux concurrents à partir de diverses ressources en ligne, qui peuvent ensuite être analysées et comparées dans l'outil.
\end{minipage}
\\*
\begin{minipage}[t]{\dimexpr(\linewidth) - 1.5em}
  \textbf{\textsc{ providentia-cataclyst_legacy }} \website{ https://github.com/lyszt/providentia-cataclyst_legacy }{ providentia-cataclyst_legacy } \hfill \textsc{ Legacy }\smallskip\\
  Cataclyst, le successeur du robot Providentia original, est né du crucible de la guerre. Providentia Type C est une machine de destruction implacable. Cet automate implacable ne défend pas seulement l'empire mais cherche activement et efface ses ennemis.
\end{minipage}
\\*
\begin{minipage}[t]{\dimexpr(\linewidth) - 1.5em}
  \textbf{\textsc{ providentia-network_legacy }} \website{ https://github.com/lyszt/providentia-network_legacy }{ providentia-network_legacy } \hfill \textsc{ Legacy }\smallskip\\
  Providentia Network: Un backend Django avec une base de données persistante, exécutant simultanément des robots Discord et Telegram.
\end{minipage}
\\*
\begin{minipage}[t]{\dimexpr(\linewidth) - 1.5em}
  \textbf{\textsc{ l'héritage providentiel de type d }} \website{ https://github.com/lyszt/providentia-type-d_legacy }{ l'héritage providentiel de type d } \hfill \textsc{ Legacy }\smallskip\\
  Providentia - Type D est un robot Discord sophistiqué développé par les esprits visionnaires de l'Empire Kaiser.
\end{minipage}
\\*
\begin{minipage}[t]{\dimexpr(\linewidth) - 1.5em}
  \textbf{\textsc{ l'héritage de la famille }} \website{ https://github.com/lyszt/scarlett-dash_legacy }{ l'héritage de la famille } \hfill \textsc{ Legacy }\smallskip\\
  Le tableau de bord personnel.
\end{minipage}
\\*
\begin{minipage}[t]{\dimexpr(\linewidth) - 1.5em}
  \textbf{\textsc{ l'héritage de la porte de scarlett }} \website{ https://github.com/lyszt/scarlett-gateway_legacy }{ l'héritage de la porte de scarlett } \hfill \textsc{ Legacy }\smallskip\\
  
\end{minipage}
\\*
\end{projects}