% Section: Projects (from repos or config; no tags)
\sectionTitle{ Projekter }{\faCodeFork}

\begin{projects}
\begin{minipage}[t]{\dimexpr(\linewidth) - 1.5em}
  \textbf{\textsc{ autoresume_core }} \website{ https://github.com/lyszt/autoresume_core }{ autoresume_core } \hfill \textsc{ Core }\smallskip\\
  
\end{minipage}
\\*
\begin{minipage}[t]{\dimexpr(\linewidth) - 1.5em}
  \textbf{\textsc{ capitalismo-automatique_core }} \website{ https://github.com/lyszt/capitalisme-automatique_core }{ capitalismo-automatique_core } \hfill \textsc{ Core }\smallskip\\
  Uma sequência de um simulador de sociedade java feito durante uma aula de Java, mas em Python
\end{minipage}
\\*
\begin{minipage}[t]{\dimexpr(\linewidth) - 1.5em}
  \textbf{\textsc{ clairemont_core }} \website{ https://github.com/lyszt/clairemont_core }{ clairemont_core } \hfill \textsc{ Core }\smallskip\\
  
\end{minipage}
\\*
\begin{minipage}[t]{\dimexpr(\linewidth) - 1.5em}
  \textbf{\textsc{ dannazione-di-provvidenza_core }} \website{ https://github.com/lyszt/dannazione-di-provvidenza_core }{ dannazione-di-provvidenza_core } \hfill \textsc{ Core }\smallskip\\
  Dannazione di Providenza é um assistente de aprendizado de idiomas com tecnologia de inteligência artificial com recursos OCR, serviços de tradução e uma extensão do navegador Firefox.
\end{minipage}
\\*
\begin{minipage}[t]{\dimexpr(\linewidth) - 1.5em}
  \textbf{\textsc{ - O que é que é? }} \website{ https://github.com/lyszt/french-ship-madame-de-pompadour_core }{ - O que é que é? } \hfill \textsc{ Core }\smallskip\\
  Simulação social e gerador de histórias que cria narrativas de ficção científica baseadas em Star Trek. OOP Python/Flask com algumas estruturas de dados, Gemini API, React/Vite/Tailwind frontend.
\end{minipage}
\\*
\begin{minipage}[t]{\dimexpr(\linewidth) - 1.5em}
  \textbf{\textsc{ iris-client_core }} \website{ https://github.com/lyszt/iris-client_core }{ iris-client_core } \hfill \textsc{ Core }\smallskip\\
  Interface de utilizador do terminal feita para o projeto de gerenciador de projetos IRIS.
\end{minipage}
\\*
\begin{minipage}[t]{\dimexpr(\linewidth) - 1.5em}
  \textbf{\textsc{ - O que é isso? }} \website{ https://github.com/lyszt/kyrie_core }{ - O que é isso? } \hfill \textsc{ Core }\smallskip\\
  
\end{minipage}
\\*
\begin{minipage}[t]{\dimexpr(\linewidth) - 1.5em}
  \textbf{\textsc{ Providence_core }} \website{ https://github.com/lyszt/providence_core }{ Providence_core } \hfill \textsc{ Core }\smallskip\\
  Rede de Providentia Refactored.
\end{minipage}
\\*
\begin{minipage}[t]{\dimexpr(\linewidth) - 1.5em}
  \textbf{\textsc{ providentia-access_core }} \website{ https://github.com/lyszt/providentia-access_core }{ providentia-access_core } \hfill \textsc{ Core }\smallskip\\
  Gerente de Password e Authentication da Providentia.
\end{minipage}
\\*
\begin{minipage}[t]{\dimexpr(\linewidth) - 1.5em}
  \textbf{\textsc{ Providentia-magnata_core }} \website{ https://github.com/lyszt/providentia-magnata_core }{ Providentia-magnata_core } \hfill \textsc{ Core }\smallskip\\
  A Providence é um bot Discord projetado para analisar mensagens de usuários, gerenciar dados de usuários e melhorar o engajamento através da tradução de linguagem e análise de sentimentos.
\end{minipage}
\\*
\begin{minipage}[t]{\dimexpr(\linewidth) - 1.5em}
  \textbf{\textsc{ Ryujinni_core }} \website{ https://github.com/lyszt/ryujinni_core }{ Ryujinni_core } \hfill \textsc{ Core }\smallskip\\
  A versão Elixir do bot Clairemont da Discord, alimentado pela Providence Network (Providentia).
\end{minipage}
\\*
\begin{minipage}[t]{\dimexpr(\linewidth) - 1.5em}
  \textbf{\textsc{ Scarlett-Citadel_core }} \website{ https://github.com/lyszt/scarlett-citadel_core }{ Scarlett-Citadel_core } \hfill \textsc{ Core }\smallskip\\
  A Cidadela Scarlett, a versão remodelada do meu site pessoal, compilada usando o vite num site estático, mudada para Tesserae, de facto.
\end{minipage}
\\*
\begin{minipage}[t]{\dimexpr(\linewidth) - 1.5em}
  \textbf{\textsc{ número de base }} \website{ https://github.com/lyszt/talleyrand_core }{ número de base } \hfill \textsc{ Core }\smallskip\\
  .config Linux configuration manager construído em C++.
\end{minipage}
\\*
\begin{minipage}[t]{\dimexpr(\linewidth) - 1.5em}
  \textbf{\textsc{ Tesserae_core }} \website{ https://github.com/lyszt/tesserae_core }{ Tesserae_core } \hfill \textsc{ Core }\smallskip\\
  Retour de pensamento baseado em gato lyszt.github.io com um servidor para engajamento comunitário.
\end{minipage}
\\*
\begin{minipage}[t]{\dimexpr(\linewidth) - 1.5em}
  \textbf{\textsc{ trabalho-ed2_edu }} \website{ https://github.com/lyszt/trabalho-ed2_edu }{ trabalho-ed2_edu } \hfill \textsc{ Education }\smallskip\\
  
\end{minipage}
\\*
\begin{minipage}[t]{\dimexpr(\linewidth) - 1.5em}
  \textbf{\textsc{ trabalho }} \website{ https://github.com/lyszt/workonomia_edu }{ trabalho } \hfill \textsc{ Education }\smallskip\\
  A Workonomia é um simulador de gestão econômica desenvolvido em Java usando a biblioteca Swing, focado na gestão de recursos e otimização do sistema.
\end{minipage}
\\*
\begin{minipage}[t]{\dimexpr(\linewidth) - 1.5em}
  \textbf{\textsc{ Kairos-praxis_legacia }} \website{ https://github.com/lyszt/kairos-praxis_legacy }{ Kairos-praxis_legacia } \hfill \textsc{ Legacy }\smallskip\\
  Kairos Praxis (Insight 2) oferece funcionalidades de análise de concorrentes através de uma aplicação de desktop.
\end{minipage}
\\*
\begin{minipage}[t]{\dimexpr(\linewidth) - 1.5em}
  \textbf{\textsc{ Providentia-cataclyst_legacia }} \website{ https://github.com/lyszt/providentia-cataclyst_legacy }{ Providentia-cataclyst_legacia } \hfill \textsc{ Legacy }\smallskip\\
  Cataclyst, o sucessor do original bot Providentia, nasceu do crucível da guerra. Providentia Tipo C é uma máquina implacável de destruição.
\end{minipage}
\\*
\begin{minipage}[t]{\dimexpr(\linewidth) - 1.5em}
  \textbf{\textsc{ Providentia-network_legacia }} \website{ https://github.com/lyszt/providentia-network_legacy }{ Providentia-network_legacia } \hfill \textsc{ Legacy }\smallskip\\
  Providentia Network: Um backend com Django com um banco de dados persistente, executando simultaneamente bots Discord e Telegram. Integra com serviços externos como Google e sistemas acadêmicos universitários para raciocínio e ações avançadas.
\end{minipage}
\\*
\begin{minipage}[t]{\dimexpr(\linewidth) - 1.5em}
  \textbf{\textsc{ Providentia-type-d_legacy }} \website{ https://github.com/lyszt/providentia-type-d_legacy }{ Providentia-type-d_legacy } \hfill \textsc{ Legacy }\smallskip\\
  Providentia - Tipo D é um sofisticado bot Discord desenvolvido pelas mentes visionárias do Império Kaisaran. Originalmente projetado como uma arma militar para operações estratégicas, este bot passou por uma transformação, evoluindo para uma ferramenta de automação versátil e um guardião confiável para a segurança do servidor.
\end{minipage}
\\*
\begin{minipage}[t]{\dimexpr(\linewidth) - 1.5em}
  \textbf{\textsc{ Scarlett-dash_legacy }} \website{ https://github.com/lyszt/scarlett-dash_legacy }{ Scarlett-dash_legacy } \hfill \textsc{ Legacy }\smallskip\\
  Painel pessoal.
\end{minipage}
\\*
\begin{minipage}[t]{\dimexpr(\linewidth) - 1.5em}
  \textbf{\textsc{ Scarlett-gateway_legacy }} \website{ https://github.com/lyszt/scarlett-gateway_legacy }{ Scarlett-gateway_legacy } \hfill \textsc{ Legacy }\smallskip\\
  
\end{minipage}
\\*
\end{projects}